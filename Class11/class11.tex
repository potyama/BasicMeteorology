\documentclass{jsarticle}
\oddsidemargin=-0.8cm
\topmargin=-2cm
\baselineskip=13pt
\textheight=55\baselineskip
\marginparsep=0.5in
\marginparwidth=0.5in
\textwidth=52zw
\usepackage{ascmac}
\usepackage{url}
\usepackage{here}
\usepackage[dvips]{graphicx}
\usepackage{amsmath}
\usepackage{amssymb}
\usepackage{multicol}
\usepackage{bm}
\usepackage{enumerate}
\usepackage{listings}
\usepackage{fancybox}
\usepackage{framed}
\usepackage{subfigure}
\usepackage{ccaption}
\usepackage{color}
\makeatletter
\lstset{% 
language={C}, 
frame=trbl,% 
basicstyle={\small},% 
identifierstyle={\small},% 
commentstyle={\small\ttfamily},% 
keywordstyle={\small\bfseries},% 
ndkeywordstyle={\small},% 
stringstyle={\small\ttfamily}, 
tabsize=2,
breaklines=true, 
frame=none,
columns=[l]{fullflexible},% 
numbers=left,% 
xrightmargin=0zw,% 
xleftmargin=3zw,% 
numberstyle={\scriptsize},% 
stepnumber=1, 
numbersep=1zw,% 
backgroundcolor={\color[gray]{.90}},
} 
\makeatother

\newenvironment{problems}
{
  \renewcommand\labelenumi{\doublebox{\arabic{enumi}}}
  \begin{enumerate}
}{
  \end{enumerate}
  \renewcommand\labelenumi{\arabic{enumi}.}
}

\pagestyle{empty}	

\begin{document}
\title{基礎気象学講義 復習問題} % ここは毎回同じ
\author{第11回} %authorの代わりに第何回かを入れる
\date{メソ気象学} %内容を記載する
\maketitle

\section{問題}

    \begin{problems}
    \item 夏を中心とした暑い日の夕方、対流雲が立ち上がり、雷雨となることがある。いわゆる「夕立」である。これについて、次の問いに答えなさい。
        \begin{enumerate}[(1)]
        \item 晴れた夏の日が夕立に至るまでの成因を述べなさい。
        \item 夕立が起きやすいと考えられる気象的・地形的条件を3つ程度挙げなさい。
        \item 日本国内において、具体的にどの地域で夕立が起きやすいのか、何箇所か例を挙げなさい。また、そのうちの1例について、実際に夕立が起きていることが見いだせるように観測データを示しなさい。
        \item 気象的条件から夕立を予測する場合、どのような点に着目すればよいか、先の気象的条件をもとに述べなさい。
        \item これら対流雲による気象災害は、気象庁の発表するどのような情報により注意喚起されるか、情報の名称を2例程度挙げなさい。
        \item 対流雲による気象災害から自身の身を守るためにどのように行動すればよいか、まとめなさい。\\
        \end{enumerate}

    \item 海陸風(湖陸風などでも良い)や山谷風が見られると考えられるアメダス観測点を1点選んで、これを観察してみましょう。
        \begin{enumerate}[(1)]
        \item その周辺の地形図を示し、理論通りに吹くならば風がどのように変化するか説明しなさい。
        \item 当該地点の適当なデータを選択し、にホドグラフをプロットしなさい。ただし、ホドグラフは時間変化がわかるようにプロットしなさい。
        \item 時間軸を横軸、風向を縦軸にとって、風向の時間変化を示したグラフを作成しなさい。但し、風向は海陸風等の動径方向と垂直な角度に0度をとり、動径方向が正負になるよう、絶対値が180度以下の範囲で表すこと。
        \item 上記をもとに、観測地点の風の日変化について考察しなさい。
        \item 風の日変化は日照時間にどのように影響されているか調べなさい。特に、季節変化に留意しなさい。
        \item 風の日変化は温度など他の気象要素にどのように影響するか調べなさい。\\
        \end{enumerate}

    \item CM=5のAcは山岳波の可視化として知られている。これが山岳波によってどのように生まれ、どのように可視化しているのかを、図及び写真を用いて説明しなさい。
    \end{problems}

\section{答案}

\begin{problems}
\item
        \begin{enumerate}[(1)]
        \item 午前中に日射によって地表面の空気が温められ水蒸気が蒸発する.このため,地表付近の空気が上昇気流を発生させる.また,地形条件や気象条件によって地表付近の水蒸気が上空に強く持ち上げられる.積乱雲が発生する.この積乱雲が夕立を発生させる.
        \item 気象的条件:上昇気流,大気不安定,地形的条件:斜面,ヒートアイランド現象が発生するような都市部
        \item 都心部だと思うが,観測データ見つからず
        \item 高層天気図を見て大気の安定度を見る.地上天気図で前線が近くにないか確認する.
        \item 雷注意情報,竜巻注意情報
        \item 積乱雲を見かけたら外出を控える.屋内に早めに避難する.また,気象庁による注意報・警報を確認する.
        \end{enumerate}
 \item
        \begin{enumerate}[(1)]
        \item 今回は、群馬県前橋市を選択した。前橋市の周辺地形図は図\ref{fig:maebashi}のようになっている。
        \begin{figure}[ptbh]
        \centering  % 図を真ん中に配置
        \includegraphics[width=0.5\linewidth, keepaspectratio]{maebashi.eps}
        \caption{前橋市の周辺地形図}
        \label{fig:maebashi}
        \end{figure}
        図\ref{fig:maebashi}から、前橋市に入るまでは太平洋側から、すなわち北西の風が吹き前橋市を超えたあたりから山脈に囲まれているため、北風が吹くと予想される。
        \item  hodogpaph.pngに結果を示す。
        \item wind.pngに風向の時間変化を示す。
        \item まず、早朝をみると、東風が多いことがわかる。すなわち、太平洋側に向かって吹いており、陸風が吹いていることがわかる。一方、午前中は北風に変わっている。これは(1)で予想した風向きと同じで海風であると考える。午後になると、風向きが安定していない。これは、太陽による放射熱や地形の影響によりこのような風向きになっていると考える。
        \item 日照時間が長いと、風向や風速が強くなると考えられる。放射熱により、地表面が温まり、上昇気流が発生する。特に夏は日照時間が長いため、風向や風速が安定しないと考えられる。
        \item 例えば、日照時間が長いと、強い上昇気流が発生する一つの原因となり、積乱雲が発達し、夕立などといった急な雨が降る可能性がある。
        \end{enumerate}
\end{problems}

\section{読書案内}
この分野は邦書が十分にあるので、邦書のみを挙げる。
\begin{itemize}
\item 小倉義光 1997 "メソ気象の基礎理論" 東京大学出版会
\item 吉崎正憲,加藤輝之 2007 "豪雨・豪雪の気象学" 朝倉書店
\item 斉藤和雄,鈴木修 2016 "メソ気象の監視と予測" 朝倉書店
\item 加藤輝之 2022 "集中豪雨と線状降水帯" 朝倉書店
\item 加藤輝之 2017 "図解説 中小規模気象学" 気象庁 (\url{https://www.jma.go.jp/jma/kishou/know/expert/pdf/textbook_meso_v2.1.pdf})
\end{itemize}

\end{document}
