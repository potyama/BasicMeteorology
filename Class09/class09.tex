\documentclass{jsarticle}
\oddsidemargin=-0.8cm
\topmargin=-2cm
\baselineskip=13pt
\textheight=55\baselineskip
\marginparsep=0.5in
\marginparwidth=0.5in
\textwidth=52zw
\usepackage{ascmac}
\usepackage{url}
\usepackage[dvips]{graphicx}
\usepackage{amsmath}
\usepackage{amssymb}
\usepackage{multicol}
\usepackage{bm}
\usepackage{enumerate}
\usepackage{listings}
\usepackage{fancybox}
\usepackage{framed}
\usepackage{subfigure}
\usepackage{ccaption}
\usepackage{color}
\makeatletter
\lstset{% 
language={C}, 
frame=trbl,% 
basicstyle={\small},% 
identifierstyle={\small},% 
commentstyle={\small\ttfamily},% 
keywordstyle={\small\bfseries},% 
ndkeywordstyle={\small},% 
stringstyle={\small\ttfamily}, 
tabsize=2,
breaklines=true, 
frame=none,
columns=[l]{fullflexible},% 
numbers=left,% 
xrightmargin=0zw,% 
xleftmargin=3zw,% 
numberstyle={\scriptsize},% 
stepnumber=1, 
numbersep=1zw,% 
backgroundcolor={\color[gray]{.90}},
} 
\makeatother

\newenvironment{problems}
{
  \renewcommand\labelenumi{\doublebox{\arabic{enumi}}}
  \begin{enumerate}
}{
  \end{enumerate}
  \renewcommand\labelenumi{\arabic{enumi}.}
}

\pagestyle{empty}	

\begin{document}
\title{基礎気象学講義 復習問題} % ここは毎回同じ
\author{第9回} %authorの代わりに第何回かを入れる
\date{気象力学の重要な応用:数値予報} %内容を記載する
\maketitle

\section{問題}

    \begin{problems}
    \item 次の文章を読んで、後の問いに答えなさい。
        \begin{screen}
        $_{(\mathrm{a})}$\underline{気象力学の知見により得られた方程式}を、現況を初期値として数値的に積分することにより、将来の気象状況を予報できる。
        これが、気象力学の重要な応用として知られる数値予報である。

        気象力学の方程式には、時間発展を含まない( A )と、時間発展を含む( B )がある。
        観測された気象要素から、他の気象要素を計算して現況を求める場合などには(A)が、将来の時間の予報を行う場合には(B)を用いる。

        これらを数値的に解く場合には、連続な値としては扱えないので、適当に格子を設定して扱う。
        もっとも、観測地点が格子状に配列されているわけでもなく、上空の状況を見られるのは尚限られるため、実際には現況から格子状の値を推定しなければならない。
        この過程を( C )と呼び、得られた結果を( D )値と呼ぶ。(C)の手法としては、3次元や4次元の変分法、最適内挿法などが知られている。

        格子状の初期値が得られれば、今度はこれを元に積分計算を行う。
        この手法には、微分方程式を$_{(\mathrm{b})}$\underline{離散化して漸化式のようにみなして計算}する( E )や、解のスペクトルを数値的に定めて波動の発展により予報を行う( F )がある。

        首尾よく計算された結果は一つの予報値であり、これを予報者が読める形に直せば数値予報のプロダクトとなるが、これには幾ばくかの誤差が含まれている。
        この誤差傾向は人間が解析している他、統計的な手法(深層学習などもこの範疇に属する)を用いて修正も行うことができる。
        修正を行った$_{(\mathrm{c})}$\underline{数値予報プロダクト}を( G )と呼ぶ。
        (G)の予報は現状比較的確かなものであるが、統計に無い例の激甚な災害のこともありうることから、これを最終的な予報とすべきかどうかには、確かな知見を持った人間の判断を要する。
        実際には、$_{(\mathrm{d})}$\underline{計算結果が出るまでの時間}などの細かい現況と、プロダクトのズレから、人間が修正を加える場合が少なくない。

        前述の誤差には、数値計算の近似による誤差と観測値あるいはそこからの推定に伴う初期値の誤差が混在している。
        一方、気象現象はカオスとして知られており、僅かな誤差の差異が後の計算結果に大きく影響を与える。
        そこで、初期値にある一定のパターンの誤差(摂動)が含まれると仮定し、この誤差を修正した複数の初期値から計算を行う、$_{(\mathrm{e})}$\underline{ある種乱択的な手法}も便利である。
        この手法のことを( H )と呼ぶ。これに対し、初期値の誤差を無視し、そのまま計算した結果のみを見る手法を( I )と呼ぶ。
            \begin{flushright}
            (大関「スペクトルモデル入門」・小倉「お天気の科学」を参考に作成)
            \end{flushright}
        \end{screen}

        \begin{enumerate}[(1)]
        \item 文中の空欄(A)〜(I)に当てはまる語句や数値を答えなさい。
        \item 下線部(a)について、気象力学の出発点となる5つの方程式を総称してなんというか答えなさい。また、これを(A)と(B)に分類しなさい。
        \item 下線部(b)について、その計算法を2例挙げ、近似による誤差がどの程度になるかを求めなさい。
        \item 下線部(c)について、数値予報の結果として公開されているプロダクトの例を2例程度挙げなさい。
        \item 下線部(d)について、計算結果が出るまでに時間を要する原因を複数挙げなさい。
        \item 下線部(e)について、この計算結果として一般に広く使われている気象資料を1例挙げなさい。
        \item 気象庁において、短期の(I)を行う数値予報モデルとしてはGSM・MSM・LFMが知られている。これらの諸元について調べ、表にまとめなさい。\\
        \end{enumerate}

    \item 講義ではコリオリ力の影響として慣性振動のシミュレーションを行った。ここでは、別の例としてフーコーの振り子について考察してみよう。
    今、十分長く、伸縮しない糸により、単位質量の質点がぶら下げられており、振り子の運動をする。この時、この振り子の振れ幅は小さく、高さ($z$)方向への変動は無視できる程度とする。
    この振り子を適当な位置に持っていった後、手を放した。この位置を$(x_0,0)$として(但し$x_0>0$)、シミュレーションを行おう。
    \begin{enumerate}[(1)]
    \item 振り子が点$(x,y)$にある時の$x,y$各方向の運動方程式を立てなさい。但し、$\sqrt{x^2+y^2}\le x_0$とする。
    \item 運動方程式について、Euler法及びLeap-Frog法により差分化した式を記しなさい。
    \item Euler法、Leap-Frog法により、12時間程度の間の位置の変化を求め、グラフで記載しなさい。時間ステップは任意に定めて構いません。
    \item Euler法、Leap-Frog法により、12時間程度の間の運動エネルギーの変化を求め、グラフで記載しなさい。時間ステップは任意に定めて構いません。\\
    \end{enumerate}

    \item 気象現象はカオスであると説明したが、実際にカオスとはどのような現象であるのか、エノン写像という例で計算し、確認してみよう。エノン写像は、以下の漸化式により与えられる。
    \[
    \left\{
    \begin{array}{l}
    x_{n+1}=1-ax_n^2+y_n \\
    y_{n+1}=bx_n
    \end{array}
    \right.
    \]
    ここで、$a,b$は定数である。
    \begin{enumerate}[(1)]
    \item $b=0.3$で固定させ、$a$を0.9から1.45の間で0.01ずつ変化させる(56種類のデータとなる)。このときのグラフを描き、いくつかの$n$について、「すべての点を内側に含む円」を描きなさい。
    \item $a=1.4$で固定させ、$b$を0.2から0.4の間で0.01ずつ変化させる(21種類のデータとなる)。このときのグラフを描き、いくつかの$n$について、「すべての点を内側に含む円」を描きなさい。
    \item 「すべての点を内側に含む円」の半径は$a,b$に対してどの様に推移していくかグラフにプロットして示しなさい。また、これをいくつかの$n$について行うことで、「先の予報ほど不確かになる」理由を確かめなさい。
    \end{enumerate}

\end{problems}

\section{答案}
\begin{problems}
% 以下に解答を作成してGit Push。
\item 

\item 

\item 

\end{problems}

\section{読書案内}
数値予報は華形と言って差し支えない分野であり、テクノロジー面なども含め一般に興味を持ってもらいやすい。このことから、数値予報に関する本はもちろん、関係する教科書に数値予報の話が入っていることも多い。

以下、数値予報を主体に書かれているものを示す。
\begin{itemize}
\item 気象庁 "数値予報解説資料集"
\item 気象庁 "数値予報研修テキスト"
\item 気象庁 "数値予報60年誌"
\item 古川武彦,室井ちあし 2012 "現代天気予報学" 朝倉書店
\item 新田尚,二宮洸三,山岸米二郎 2009 "数値予報と現代気象学" 東京堂出版
\item 二宮洸三 2004 "数値予報の基礎知識" オーム社
\item 時岡達志,山岬正紀,佐藤信夫 1993 "気象の数値シミュレーション" 東京大学出版会
\item 大関誠 2006 "スペクトルモデル入門" 日本気象学会
\item 露木義,川畑拓矢 2008 "気象学におけるデータ同化" 日本気象学会
\item 淡路敏之,池田元美,石川洋一,蒲地政文 2009 "データ同化" 京都大学学術出版会
\end{itemize}

ここで紹介したのは数値予報を直接の対象としている書籍を基本としたが、数値流体力学・計算物理・数値計算・データ同化などの分野の書籍も当然数値予報の理解には役立つ。
また、諸外国においても、例えばECMWF(ヨーロッパ中期予報センター)は新任開発者向けの研修資料を公開しているなど、資料は枚挙に暇がない。

なお、原理原則などの理解はやや古い教科書でも問題ないが、実際に動いているモデルの現況などは日進月歩であることから、これら諸元などを確認する場合は新しい資料を活用することをおすすめする。

\end{document}

