\documentclass{jsarticle}
\oddsidemargin=-0.8cm
\topmargin=-2cm
\baselineskip=13pt
\textheight=55\baselineskip
\marginparsep=0.5in
\marginparwidth=0.5in
\textwidth=52zw
\usepackage{ascmac}
\usepackage{url}
\usepackage[dvips]{graphicx}
\usepackage{amsmath}
\usepackage{amssymb}
\usepackage{multicol}
\usepackage{bm}
\usepackage{enumerate}
\usepackage{listings}
\usepackage{fancybox}
\usepackage{framed}
\usepackage{subfigure}
\usepackage{ccaption}
\usepackage{color}
\makeatletter
\lstset{% 
language={C}, 
frame=trbl,% 
basicstyle={\small},% 
identifierstyle={\small},% 
commentstyle={\small\ttfamily},% 
keywordstyle={\small\bfseries},% 
ndkeywordstyle={\small},% 
stringstyle={\small\ttfamily}, 
tabsize=2,
breaklines=true, 
frame=none,
columns=[l]{fullflexible},% 
numbers=left,% 
xrightmargin=0zw,% 
xleftmargin=3zw,% 
numberstyle={\scriptsize},% 
stepnumber=1, 
numbersep=1zw,% 
backgroundcolor={\color[gray]{.90}},
} 
\makeatother

\newenvironment{problems}
{
  \renewcommand\labelenumi{\doublebox{\arabic{enumi}}}
  \begin{enumerate}
}{
  \end{enumerate}
  \renewcommand\labelenumi{\arabic{enumi}.}
}

\pagestyle{empty}	

\begin{document}
\title{基礎気象学講義 復習問題} % ここは毎回同じ
\author{第6回} %authorの代わりに第何回かを入れる
\date{雲と降水の分類} %内容を記載する
\maketitle

\section{問題}

    \begin{problems}
    \item 次の問いに答えなさい。
        \begin{enumerate}[(1)]
        \item 雲は大きく10種類に分かれる(十種雲形)。これらを出る層がわかるように一覧に並べ、和名・英名・記号をまとめた表を作成しなさい。
        \item 気象観測の手引き(\url{https://www.jma.go.jp/jma/kishou/know/kansoku_guide/tebiki.pdf})に従い、次の降水現象を説明しなさい。
        \begin{itemize}
            \item 霧雨
            \item 着氷性・過冷却の霧雨
            \item 雨
            \item 着氷性・過冷却の雨
            \item 雪
            \item みぞれ
            \item 雪あられ
            \item 霧雪
            \item 凍雨
            \item 氷あられ
            \item ひょう
            \item 細氷\\
        \end{itemize}
        \end{enumerate}

    \item 昼間に空を観測し、全雲量・各層の雲の型・出ている雲形とその雲量・天気を記録しなさい。これを6時間以上あけて3度以上行うと共に、その時の空の写真を貼り付けなさい。\\

    \item International Cloud Atlas(国際雲図帳、\url{https://cloudatlas.wmo.int/en/home.html})のサイトには、WMO(世界気象機関)の提供する様々な雲に関する知見がまとめられている。
        当該サイトのImagesから適当な雲の写真を3点選び、その解説を和訳しなさい。

\end{problems}

\section{答案}
\begin{problems}
% 以下に解答を作成してGit Push。
\item 

\item 

\item 

\end{problems}

\end{document}

